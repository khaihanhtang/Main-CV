\documentclass[a4paper, 11pt]{article}
\usepackage{geometry}
\usepackage{comment}
\geometry{a4paper,
	total={10mm,10mm},
	left=10mm,
	top=10mm}
%A Few Useful Packages
\usepackage{marvosym}
\usepackage{fontspec} 					%for loading fonts
\usepackage{xunicode,xltxtra,url,parskip} 	%other packages for formatting
\RequirePackage{color,graphicx}
\usepackage[usenames,dvipsnames]{xcolor}
\usepackage[big]{layaureo} 				%better formatting of the A4 page
% an alternative to Layaureo can be ** \usepackage{fullpage} **
\usepackage{supertabular} 				%for Grades
\usepackage{titlesec}					%custom \section
\usepackage{adjustbox}
\usepackage{comment}
\usepackage{enumitem}

%Setup hyperref package, and colours for links
\usepackage{hyperref}
\definecolor{linkcolour}{rgb}{0,0.2,0.6}
\hypersetup{colorlinks,breaklinks,urlcolor=linkcolour, linkcolor=linkcolour}

%FONTS
\defaultfontfeatures{Mapping=tex-text}
%\setmainfont[SmallCapsFont = Fontin SmallCaps]{Fontin}
%%% modified for Karol Kozioł for ShareLaTeX use
\setmainfont[
SmallCapsFont = Fontin-SmallCaps.otf,
BoldFont = Fontin-Bold.otf,
ItalicFont = Fontin-Italic.otf
]
{Fontin.otf}
%%%

%CV Sections inspired by: 
%http://stefano.italians.nl/archives/26
\titleformat{\section}{\Large\scshape\raggedright}{}{0em}{}[\titlerule]
\titlespacing{\section}{0pt}{2pt}{2pt}
%Tweak a bit the top margin
%\addtolength{\voffset}{-1.3cm}

%Italian hyphenation for the word: ''corporations''
\hyphenation{im-pre-se}

%-------------WATERMARK TEST [**not part of a CV**]---------------
\usepackage[absolute]{textpos}

\setlength{\TPHorizModule}{10mm}
\setlength{\TPVertModule}{\TPHorizModule}
\textblockorigin{2mm}{1.0\paperheight}
\setlength{\parindent}{0pt}

%--------------------BEGIN DOCUMENT----------------------
\begin{document}
	
	%WATERMARK TEST [**not part of a CV**]---------------
	%\font\wm=''Baskerville:color=787878'' at 8pt
	%\font\wmweb=''Baskerville:color=FF1493'' at 8pt
	%{\wm 
		%	\begin{textblock}{1}(0,0)
			%		\rotatebox{-90}{\parbox{500mm}{
					%			Typeset by Alessandro Plasmati with \XeTeX\  \today\ for 
					%			{\wmweb \href{http://www.aleplasmati.comuv.com}{aleplasmati.comuv.com}}
					%		}
				%	}
			%	\end{textblock}
		%}
	
	\pagestyle{empty} % non-numbered pages
	
	\font\fb=''[cmr10]'' %for use with \LaTeX command
	
	%--------------------TITLE-------------
	\par{\centering
		{\Huge Khai Hanh Tang
		}\bigskip\par}
	
	%--------------------SECTIONS----------------------------------
	
	%Section: Personal Data
	\textbf{Email.} tang0404@e.ntu.edu.sg, khaihanhtang@gmail.com\\
	\textbf{Mobile.} (+65) 9469 0354
	\begin{comment}
		\section{Personal Information}
		
		\begin{tabular}{rl}
			\textsc{Placeof Birth:} & Sadec City, Dong Thap Province, Vietnam \\
			\textsc{Date of Birth:} & September 16\textsuperscript{th} 1995 \\
			\textsc{Address:}   & University of Science, VNU-HCM\\
			& 227 Nguyen Van Cu, Ward 1, District 5, Ho Chi Minh city, Vietnam \\
			\textsc{Phone:}     & (+84) 905 718 507\\
			\textsc{email:}     &  \href{mailto:tang0404@e.ntu.edu.sg}{tang0404@e.ntu.edu.sg}
		\end{tabular}
	\end{comment}
	\section{Education}
	\begin{tabular}{rl}	
		1/2018 -- Present & \textbf{\textit{PhD} Student in Cryptography} \\
		& Nanyang Technological University, Singapore   \\
		\ & \textbf{\textit{Thesis title:}} \textit{``New and Improved Zero-Knowledge Argument Systems}\\&\textit{           for Code-Based Cryptopgraphy."}\\
		
		\ & \textbf{\textit{Advisors:}} \textit{Assoc. Prof.} Huaxiong Wang, \textit{Prof.} San Ling, \textit{Dr.} Khoa Nguyen.\\
		\ & \ \\
		8/2013 -- 11/2017 & \textbf{\textit{B.Sc in Computer Science}} \\ 
		& \textit{Excellence Program}, Faculty of Information Technology,\\ &University of Science, VNU-HCM.\\
		\ & \textbf{\textit{GPA:}} 9.05/10 (or 3.75 / 4.00).
	\end{tabular}
	
	\section{Publication}
	Khoa Nguyen, \textbf{Hanh Tang}, Huaxiong Wang, Neng Zeng. ``New Code-Based Privacy-Preserving Cryptographic Constructions''. In ASIACRYPT 2019. LNCS, vol. 11922, pp. 25-55. Springer, 2019.
	
	San Ling, Khoa Nguyen, Duong Hieu Phan, \textbf{Hanh Tang}, and Huaxiong Wang. ``Zero-knowledge
	proofs for committed symmetric boolean functions''. In PQCrypto 2021,
	vol. 12841 of LNCS, pp. 339–359. Springer, 2021.
	
	\section{Research Experience}
	\textbf{In Nanyang Technological University}\\
	\textit{PhD Student}
	\begin{itemize}
		%\begin{comment}\item \textbf{New methods for designing practically efficient zero-knowledge arguments for Boolean circuits.} \begin{itemize}
		%\item Designing a post-quantum zero-knowledge argument of knowledge for proving the knowledge of a secret input which, via a publicly known Boolean circuit, produces the known output.
		%\item Achieving the computation complexity and communication cost which are competitive to the currently known constructions derived from multi-party computation.
		%\end{itemize}\end{comment}
		\item \textbf{New zero-knowledge argument systems and privacy-preserving constructions for code-based cryptography.}
		\begin{itemize}
			\item Refining Stern's framework for constructing zero-knowledge proofs/arguments of knowledge.
			\item Constructing code-based zero-knowledge arguments for inequalities of committed signed fractional numbers.
			\item Constructing code-based zero-knowledge argument/proof for correct evaluations on committed symmetric Boolean functions and committed inputs.
			\item Designing code-based fully-dynamic group and repudiable-and-claimable ring signatures.
			\item Proposing new definition of fully dynamic attribute-based signatures and realizing its construction in code-based assumptions.
		\end{itemize}
	\end{itemize}
	
	\section{Programming Skills}
	\begin{itemize}[noitemsep]
		\item \textbf{Programming Languages.} C++, Java, Python and Solidity.
		\item \textbf{Codeforces.} TangHanh (\url{https://codeforces.com/profile/TangHanh})
		\begin{itemize}
			\item \textit{Contest rating:} $2114$ (at 3:23 PM, 18-January-2021).
			\item \textit{Title:} Master.
		\end{itemize}
	\end{itemize}
	
	\section{Internship}
	\begin{itemize}
		\item Information security internship at PayPal Singapore from 18-January to 28-May-2021.
	\end{itemize}
	
	\section{Teaching Experience}
	\begin{itemize}
		\item \textbf{2017.} TA for courses of \textit{Data Structures and Algorithms}, and \textit{Probability and Statistics} in University of Sciences, VNU-HCM, in Ho Chi Minh city, Vietnam.
		\item \textbf{2019.} TA for course of \textit{Discrete Mathematics} in Nanyang Technological University, Singapore.
		\item \textbf{2020.} TA for courses of \textit{Dicrete Mathematics}, \textit{Probability} and \textit{Statistics} in Division of Mathematical Sciences, Nanyang Technological University, Singapore.
		\item \textbf{2021.} TA for courses of \textit{Discrete Mathematics} and \textit{Groups \& Symmetries} in Division of Mathematical Sciences, Nanyang Technological University, Singapore.
	\end{itemize}
	
	\section{Relevant experience}
	\begin{tabular}{rl}
		{\sl \textbf{Member}} & HCMUS-The Wizards (ACM ICPC team), regional contest 2013,\\
		& Phuket and Da Nang sites.\\
		{\sl \textbf{Member}} & HCMUS-Lattis (ACM ICPC team), regional contest 2016,\\
		& Bangkok and Nha Trang sites. 
	\end{tabular}
	
	
	% \section{Seminar Attending}
	% Weekly, intensively reading seminar in \textit{``Computational Number Theory, Graph Theory and Cryptography''} coordinated by \textit{Dr.} Nguyen An Khuong at University of Technology, VNU-HCM.
	
	% \section{Related textbooks read}
	%     \textit{``A computational introduction to number theory and algebra''} by V. Shoup.\\
	%     \textit{``Introduction to cryptography with coding theory''} by W. Trappe, et al.\\
	%     \textit{``Expander Families and Cayley Graphs: A Beginner's Guide"} by M. Krebs, et al.\\
	%     \textit{``A first course in abstract algebra: with applications"} by M. Krebs, et al.\\
	%     \textit{``Elementary number theory, group theory and Ramanujan graphs"} by G. Davidoff, et al.\\
	%     \textit{``Zeta functions of graphs: a stroll through the garden"} by A. Terras.\\
	%     \textit{``Representation theory of finite groups: an introductory approach"} by B. Steinberg.\\
	%     \textit{``Introduction to Analytic Number Theory"} by T. Apostol.
	
	%Section: Scholarships and additional info
	\section{Scholarships and Certificates}	
	\begin{tabular}{rl}
		{\sl \textbf{AcRF Research Scholarship}} & For PhD study from 1/2018-1/2022, \\
		& School of Physical and Mathematical Sciences, NTU.\\
	\end{tabular}
	
	
	%Section: Awards
	
	\section{Awards}
	\begin{tabular}{rl}
		{\sl \textbf{NSUCrypto Diploma}}& 2017, International Students' Olympiad in Cryptography.\\
		{}& 2018, 2nd Round with Neng Zeng and Hien Chu.\\
		{}& 2019, 2nd Round with Phuong Pham and Yi Tu.\\
		{\sl \textbf{2nd Prize}}& ACM ICPC Asia Regional Contest, Nha Trang, December 2016.\\
		{\sl \textbf{1st Prize}}& ACM ICPC Vietnam National Contest, October 2016.\\
		{\sl \textbf{Honorable Mention}}& ACM ICPC Asia Regional Contest, Bangkok, November 2016.\\
		{\sl \textbf{2nd Prize}}& ACM ICPC Vietnam Northern Provincial Contest, October 2016.\\
		{\sl \textbf{1st Prize}}& ACM ICPC Vietnam Southern Provincial Contest, October 2016.\\
		{\sl \textbf{3rd Prize}}& Vietnam Olympiad in Informatics Competition\\& for High School Students, January 2012.
	\end{tabular}
	
	\section{Interests and Activities}
	Swimming, walking, cycling, skating, watching movies, and reading books.
	%\hfill
	%%\hspace{1.5cm}
	
\end{document}
